\hypertarget{modus}{}
\section{Modus operandi}
\label{sec:modus}


The stellar evolution code \texttt{ev} is designed to be a binary-evolution code.  However, it can be used to
compute the evolution of both single and binary stars, and for binaries, there are several possible modes to
use the code in.  These modes are set by two parameters in \texttt{init.run}: \hyperlink{isb}{ISB} (1 or 2,
depending on whether we want to evolve one or two stars) and \hyperlink{ktw}{KTW} (1 or 2 for non-simultaneous 
or non-simultaneous binary mode, respectively).


\subsection{Single stars}

Since \texttt{ev} is a binary-evolution code, single stars are in effect in a binary.  Since you don't want to 
waste your undoubtedly valuable (CPU) time on computing the secondary, we set \texttt{ISB}=1 (single star) and
\texttt{KTW}=1 (non-TWIN mode --- I'm not sure whether it matters, but it seems a safe way to go).  

However, here we only tell the code not to compute a model of the secondary.  It will still exist, as a point
mass.  The important thing is that Roche lobes will still be defined, and it is important to set the initial orbital
period (\hyperlink{per}{PER}, in \texttt{init.run}) to a sufficiently high value to make sure your `single star'
will not fill its Roche lobe.  In addition, I usually set \hyperlink{bms}{BMS} to twice \hyperlink{sm}{SM}, to 
make sure you don't end up with a negative secondary mass.  Experienced users may want to switch off the equations 
that govern orbital evolution, mass transfer, etc., in \texttt{init.dat}, see Sect.\,\ref{sec:initdat:eqvarbc}.



\subsection{Binary stars}

When computing the evolution of a binary, we can choose whether we want to compute a full model of the secondary,
or regard it as a point mass (which can be useful when dealing with WD, NS or BH accretors).  If we want to 
compute a detailed model of the secondary, we can choose between \emph{non-simultaneous evolution}, in which the primary
is evolved for \hypertarget{kp}{KP} timesteps, before switching to the secondary to catch up in age with the primary,
and \emph{simultaneous evolution} (also known as \emph{TWIN mode}), in which both components are evolved at the same
time, and mass transfer is taken into account implicitly (this is necessary if \emph{e.g.}\ both stars have winds).


\subsubsection{Primary + compact companion (point mass)}

Evolving a binary with a point mass is essentially similar to single-star mode, except that we will set the binary
mass (\hyperlink{bms}{BMS}) and the orbital period (\hyperlink{per}{PER}) to the values we want, and we make sure 
that exactly one of \hyperlink{cmt}{CMT} or \hyperlink{cms}{CMS} is non-zero when using a version of \texttt{init.dat}
for single-star evolution.  You may want to check whether the equations for orbital evolution and mass transfer are 
being solved, see Sect.\,\ref{sec:initdat:eqvarbc}, but in principle this is not necessary.
We also set \hyperlink{isb}{ISB}=1 (evolve one star) and \hyperlink{ktw}{KTW}=1 (non-simultaneous mode) in \texttt{init.dat}.


\subsubsection{Two components, non-simultaneous}

This mode was the original way of computing the evolution of a binary: the primary is evolved for \hypertarget{kp}{KP} 
timesteps, after which the code switches to the secondary, evolves it to the same age as the primary, and it keeps
alternating between the two.

This approximates binary evolution sufficiently well for many cases, but it will not when the secondary has a 
non-negligible wind, or when the secondary fills it's Roche lobe.  In other words, all changes to the orbit are
made by the primary and the secondary cannot have any influence on the orbit (since if it would, this would affect
the evolution of a Roche-lobe-filling primary, which has already been established in the previous semi-cycle).

During the first semi-cycle, while evolving the primary, data on orbital evolution and mass transfer are stored in 
\hyperlink{io12}{\texttt{file.io12}}, which are then read again during the second semi-cycle, where the secondary is evolved.

In order to use this mode, set \hyperlink{isb}{ISB}=2 (evolve two stars) and \hyperlink{ktw}{KTW}=1 (non-simultaneous mode),
and make sure you solve equations for mass transfer and orbital evolution (see Sect.\,\ref{sec:initdat:eqvarbc}).



\subsubsection{Two components, simultaneous (TWIN mode)}

TWIN mode was developed by Peter Eggleton as an improvement of the non-simultaneous evolution in the previous section.
It allows mass loss and mass transfer from the secondary, and, in particular, contact binaries (at least in principle).
Both stars are evolved simultaneously, and mass transfer is solved implicitly.  

In order to use TWIN mode, set \hyperlink{isb}{ISB}=2 (evolve two stars) and \hyperlink{ktw}{KTW}=2 (simultaneous mode),
and make sure you solve all necessary equations (see Sect.\,\ref{sec:initdat:eqvarbc}).



\subsection{Creating grids in mass, mass ratio and period}

\remark{This is currently broken, due to the way (output?) files are opened.}

Apart from computing the model of a single binary (or one single star), the code can be used to compute grids of models,
for ranges of initial primary mass, mass ratio and orbital period.  

When you want to compute a grid, the initial binary parameters \hyperlink{sm}{\texttt{SM}},
\hyperlink{bms}{\texttt{BMS}} and \hyperlink{per}{\texttt{PER}} in \texttt{init.run} must be set to negative 
values, to ensure that they don't override the grid values.

If you want to compute only one model, make sure \texttt{KML} = \texttt{KQL} = \texttt{KXL} = 1.

See Section~\ref{sec:initrun_grids} for more details.
