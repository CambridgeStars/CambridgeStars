\hypertarget{initrun}{}
\section{init.run}
\label{sec:initrun}

The \texttt{init.run} file contains parameters that control how to start and stop the run. 
You have to decide on each of four options, each giving two alternatives. The options are
\begin{description}
\item[a)] single stars or binary stars
\item[b)] new, i.e. starting from scratch (ZAMS), or old, e.g. starting from the end
of a previous run.
\item[c)] independent evolution (`normal mode'), or \emph{simultaneous} evolution (`TWIN mode'), of the components.
\item[d)] a `one-model' or `grid' run. A grid
  means several runs, one after the other (but simultaneous using the massively
  parallel version, not described here), with the three parameters of primary mass, 
  mass ratio and orbital period being cycled through. One-shot means what it says.
\end{description}

Not all 16 possibilities make sense: e.g. if you are doing TWIN evolution,
you won't want single stars. Many but not all of the remaining possibilities should 
be viable.


\subsection{Mode of operation}
\subsubsection*{ISB, KTW, IP1, IM1, IP2, IM2, KPT, KP}
\begin{description}
\item[ISB]\hypertarget{isb}{} evolve one or two stars. 
  ISB = 1 implies only one star should be computed in detail; 
  ISB = 2 evolves both components of a binary. For \emph{single} stars, you may still use the outer 
  (first) cycle for masses. The inner 2 cycles are automatically set to do only one 
  case each. The mass ratio and the period are of course virtually ignored for single stars, 
  but have to be supplied. The period should be so large that there is no danger of RLOF
  (see also Sect.\,\ref{sec:modus}).
  (\emph{e.g.}\ XL = 7.0, meaning a period of $\sim 10^7\,$d).
\item[KTW]\hypertarget{ktw}{}  1 for normal, non-simultaneous operation; 2 for TWIN mode, where both stars are solved 
  \emph{simultaneously}.  See Sect.\,\ref{sec:modus} for more detail.
\item[IP1]\hypertarget{ip1}{} the number (13 -- 16) of the file (fort.13 -- fort.16) where the
  initial model for $*1$ is to be taken from. ZAMS models are on fort.16.
\item[IM1]\hypertarget{im1}{} the sequential number of the model required on fort.IP1. This is
  computed automatically, from later data, if the ZAMS file fort.16 is used,
  so that if IP1 is 16, it doesn't matter what value you give for IM1, but
  you have to give a value.
\item[IP2]\hypertarget{ip2}{}  as IP1, but for $*2$.
\item[IM2]\hypertarget{im2}{} as IM1, but for $*2$.
\item[KPT]\hypertarget{kpt}{} the maximum number of timesteps for each component (2000 to 4000 
  for fairly complete evolution). 
  You may set KPT equal to -1 to indicate that the code should run until
  one of the termination conditions is met, in other words, the code will
  not stop when it reaches a predetermined number of timesteps.
\item[KP]\hypertarget{kp}{}  Do approximately KP of $*1$, then enough of $*2$
  to catch up with $*1$, then another $\sim$ KP of $*1$, etc, so that if $*2$ 
  breaks down before $*1$ you don't waste a lot of calculation on $*1$. 
  You will seldom get \emph{exactly} the number of timesteps that you ask for.
  For single stars, KP is set to KPT automatically.
\end{description}



\subsection{Grids of masses and periods}
\subsubsection*{ML1, DML, KML; ~QL1, DQL, KQL; ~XL1, DXL, KXL}
\label{sec:initrun_grids}

These three lines contain parameters for 3 nested loops (mass, mass ratio, and 
initial period) to be run through. Each loop has: starting value; increment; 
number of cases (1 more than the number of increments).
\begin{itemize} 
\item The first (outer) loop is:   log$_{10}$ (mass, solar units), starting at ML1, 
  increasing by steps of DML to ML1 + (KML -- 1) . DML
\item The second loop is:  log$_{10}$ (mass ratio in sense $\{$larger/smaller$\}$), 
  starting at QL1, increasing by steps of DQL to QL1 + (KQL -- 1) . DQL
\item The third (inner) loop is:   X $\equiv$ log$_{10}$(orbital period /{period necessary for 
    $*1$ to fill its Roche lobe when still on the ZAMS}), starting at XL1, increasing 
  by steps of DXL to XL1 + (KXL -- 1) . DXL.
\item If you want to compute only one model (single or binary), set KML = KQL = KXL = 1.
\end{itemize}

When a grid is computed, the initial binary parameters \hyperlink{sm}{\texttt{SM}},
\hyperlink{bms}{\texttt{BMS}} and \hyperlink{per}{\texttt{PER}} must be set to negative 
values, to ensure that they don't override the grid values.


\subsection{Rotation and eccentricity}
\subsubsection*{ROT, KR, EX}
\begin{description}
\item[ROT, KR]
  \begin{description} 
  \item[\texttt{KR} = 1:] $P_\mathrm{rot}$ for each star = rotational breakup period * $10^\texttt{ROT}$ \remark{PERC: breakup or RLOF at ZAMS?}
  \item[\texttt{KR} = 2:] $P_\mathrm{rot}$ for each star = max(1.05 * rotational breakup period, orbital period * $10^\texttt{ROT}$) \remark{PERC: breakup or RLOF at ZAMS?}
  \item[\texttt{KR} $\ge$ 3:] set $P_\mathrm{rot} = P_\mathrm{orb}$; (almost) synchronous rotation
  \end{description}
\item[EX] the initial eccentricity.
\end{description}


\subsection{Initial binary parameters}
\subsubsection*{SM, DTY, AGE, PER, BMS, ECC, P, ENC,  ~JMX}
(a.k.a.\ AX(1-8), JMX).
The AX's are optional replacements for the values of \texttt{SM}, \ldots, \texttt{ENC} that the 
code would normally pick up in fort.IP1 from some previous run, or from the ZAMS
library (\texttt{fort.16}). JMX similarly is an optional replacement
for JMOD. They are only applied if they are non-negative. Thus you can replace
only one, or several.

\begin{description}
\item[SM]\hypertarget{sm}{} Primary mass [$M_\odot$]
\item[DTY]\hypertarget{dty}{} Time step [yr]
\item[AGE]\hypertarget{age}{} Model age [yr]
\item[PER]\hypertarget{per}{} Orbital period [d -- or fraction of break-up?]
\item[BMS]\hypertarget{bms}{} Total binary mass [$M_\odot$]
\item[ECC]\hypertarget{ecc}{} Orbital eccentricity
\item[P]\hypertarget{p}{} Spin period of the primary [d]
\item[ENC]\hypertarget{enc}{}\hypertarget{enc}{} Artificial energy rate, see \hyperlink{cea}{CEA} and \hyperlink{cet}{CET} [?]
\item[JMX]\hypertarget{jmx}{} New model number (JMOD). Set to $-1$ to keep unchanged, to $0$ to set the mass of a ZAMS model 
  using the loop parameters \texttt{ML1,QL1} above, ignoring \texttt{SM}  (when using \texttt{IP1,2}=16) 
  \remark{True?} and to any positive value to start counting models at that value.
  For grids looping over primary mass and mass ratio, you \emph{must} set \texttt{JMX} to $0$!
  \remark{In some cases, when restarting an evolved model, you seem to have to set \texttt{JMX} to $>0$!}
\end{description}


\subsubsection*{START\_WITH\_RIGID\_ROTATION}
Can be \texttt{TRUE} or \texttt{FALSE}.




\subsection{Termination conditions}
\subsubsection*{UC:}
The last three lines are a set of 21 criteria (UC(1--21)) to determine when the run is to
be ended (e.g. when the age is greater than $2\times 10^{10}\,$yr), or when some special procedure should be initiated (e.g. the He-flash evasion). You'll have to
read the end of printb.f to figure them out completely.  In many cases the code is stopped by changing the termination code $JO$.

\subsubsection*{UC(1-7):}
\begin{description}
\item[UC(1):] (rlf1)  Terminate if FLR(=RLF?) (Sect.\,\ref{sec:plt}, nr 29) of star 1 exceeds this number ($JO=4$)  ~(0.1)
\item[UC(2):] (age)   Terminate if the age of the model in years exceeds this number ($JO=5$) ~(2e10)
\item[UC(3):] (LCarb) Terminate if $L_\mathrm{C} > $ this number ($JO=6$)  ~(100)
\item[UC(4):] (rlf2)  Terminate if FLR(=RLF?) (Sect.\,\ref{sec:plt}, nr 29) of star 2 exceeds this number ($JO=7$)  ~(0.2)
\item[UC(5):] (LHe)   Initiate He-flash evasion if $L_\mathrm{He} > $ this number, together with UC(6) ($JO=8$)  ~(1e3, lower for M$_*\approx 2M_\odot$)
\item[UC(6):] (rho)   Initiate He-flash evasion if $\log\,\rho_\mathrm{c} >$ this value (?), together with UC(5) ($JO=8$)  ~(5.3)
\item[UC(7):] (MCO)   Terminate if degenerate CO-core exceeds this mass, together with UC(8) ($JO=9$)  ~(1.2)
\end{description}

\subsubsection*{UC(8-14):}
\begin{description}
\item[UC(8):]  (rho)   Terminate if $\log\,\rho_\mathrm{c} >$ this value (?), together with UC(7) ($JO=9$)  ~(6.3) 
\item[UC(9):]  (mdot)  Terminate if $|\dot{M}| \,>\, \mathrm{UC}(9)*M_*/\tau_\mathrm{KH}$  ($JO=10$)  ~(3e2) 
\item[UC(10):] (XHe)   Change eps (next number) if the core Helium abundance drops below this number ~(0.15)
\item[UC(11):] (eps)   If $Y_\mathrm{core} <$ XHe (previous number), set \hyperlink{eps}{EPS} to this number.  Do not use, keep this number 1e-6!  ~(1e-6)
\item[UC(12):] (dtmin) Terminate if $\Delta\,t <$ dtmin (in seconds?) ~(1e6)
\item[UC(13):] (sm8)   The total mass the post-He-flash model should get, can also be used manually!  ~(1e3)
\item[UC(14):] (vmh8)  The He-core mass he post-He-flash model should get, can also be used manually!  ~(1e3)
\end{description}


\subsubsection*{UC(15-21):}
\begin{description}
\item[UC(15):] (XH)    If $>0$: terminate if the core H-abundance drops below this value, you can e.g. stop at the TAMS ($JO=51$) ~(0.0)
%\item[UC(16):]   
%\item[UC(17):]   
%\item[UC(18):]   
%\item[UC(19):]   
%\item[UC(20):]   
%\item[UC(21):]   
\item[UC(16--21):]   Unused
\end{description}


