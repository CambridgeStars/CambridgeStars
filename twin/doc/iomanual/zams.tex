\section{Creating a ZAMS model}

\emph{
  Note that this section is about manual meddeling with models --- you don't need this for normal operation of the code, 
  e.g.\ to change the ZAMS mass of a model, for that see the section \hyperlink{initrun}{init.run}.
If you want to create a ZAMS series, see the example run 01 in the directory} \texttt{run/01-zams/}.


\vspace*{0.5cm}

In order to create a (ZAMS) model of certain mass, or to obtain a series
of ZAMS models, one can use the \texttt{RMG} mass loss/gain parameter in
the \texttt{init.dat} file.  This parameter gives a mass loss or mass gain
that is proportional to the mass of the star.

The method is as follows:

\begin{itemize}
\item Choose an existing input model with a mass close to the desired mass

\item Set the parameter \hyperlink{cmi}{\texttt{CMI}} to the desired value (usually $\pm 5 \times 10^{-9}$)

\item Make sure the time step doesn't change (\hyperlink{ct1}{\texttt{CT1}} = \hyperlink{ct2}{\texttt{CT2}} = 1.00

\item Calculate the factor $f$ with which you want to change the mass to get from the model
you have to the model you want. (If you have 1.00~$M_\odot$ and want 1.02~$M_\odot$, $f=1.02$)

\item Calculate the approximate number of steps you need to take for a time step size 
$\mathrm{d}t_0 = 10^3$~yr and the \hyperlink{cmi}{\texttt{CMI}} above: $N_0 = \frac{\ln f}{\mathrm{CMI}~ \mathrm{d}t_0}$

\item Choose a (nice, round, but at least) integer number of steps $N \approx N_0$

\item Calculate the true time step for N steps: $\mathrm{d}t = \frac{\ln f}{\mathrm{CMI} ~N}$

\item Fill in the values for $\mathrm{d}t$ and $N$ in \hyperlink{initrun}{\texttt{init.run}}

\item Run the model for $N$ steps and check the final mass in \texttt{file.mod}
\end{itemize}

\vspace{1cm}

\[
dt~=~\frac{\ln\left(\frac{M_\mathrm{f}}{M_\mathrm{i}}\right)}{N ~~ \mathrm{CMI}}
\]

\vspace{1cm}

The Fortran program \texttt{makezams.f} (see \href{http://www.astro.uu.nl/~sluys/Eggleton/#tools}{website}) is supposed to do all the above.

\vspace{1cm}

If all goes well, you'll end up with the mass slightly off.  You can give your model the exact mass you want
by switching off the wind, put the desired mass in \texttt{init.run} and run another 10 models or so.

If the change in mass is less than expected, you may have chosen your timestep too long, so that the code does
not converge, recalculates the model with a smaller timestep and continues with this smaller timestep 
(since it is not allowed to change).

