\section{file.out\{1,2\}}

Note: this section is about \texttt{file.out1} and \texttt{file.out2}, 
not \texttt{file.out}!

During a stellar evolution run short summaries of the stellar parameters are written 
into the files \texttt{file.out1} and \texttt{file.out2}. It can be useful to watch this file while 
the code is running for example by typing \texttt{tail -f file.out1}.
This will show the last 10 lines of the \texttt{file.out1} file and refresh when \texttt{file.out1} 
changes (exit with \texttt{Ctrl-C}).
The files start with a copy of \texttt{init.dat}. The rest of the file consists of 
three different blocks of information:
\begin{description}
\item[Stellar snapshots:] summaries of the star at a certain model number, \emph{e.g.}\ 
  its mass, age, central composition, etc.,
\item[Stellar slices:] detailed summaries of the interior of the star, \emph{e.g.}\ $P$ 
  $\rho$, $T$, etc., on every mesh point in the star,
\item[Convergence info:] information on the convergence of the set of differential equations 
  for each iteration.
\end{description}

How often these blocks of information are printed to the file can be set with the
parameters \texttt{\hyperlink{kt1}{KT1}} -- \texttt{\hyperlink{kt5}{KT5}} in \texttt{init.dat}.



\subsection{Stellar snapshots}

\subsubsection*{Line 1:}
\begin{description}
\item[M:] Stellar mass [$M_\odot$]
\item[Porb:] Orbital period [days]
\item[xi:] Mass transfer rate [$M_\odot$ yr$^{-1}$]
\item[tn:] Nuclear timescale
\item[LH:] Luminosity by Hydrogen burning
\item[P(cr):]
\item[McHe:] Mass of Helium core
\item[CXH:] Central H Abundance
\item[CXHe:] Central He abundance
\item[CXC:] Central C abundance
\item[CXN:] Central N abundance
\item[CXO:] Central O abundance
\item[CXNe:] Central Ne Abundance
\item[CXMg:] Central Mg abundance
\item[Cpsi:] Central value of the electron degeneracy parameter
\item[Crho:] Log Central density
\item[CT:] Log Central Temperature
\end{description}



\subsubsection*{Line 2:}
\begin{description}
\item[dty:] Time step [yr]
\item[Prot:] Rotational period [days]
\item[zet:] Mass loss rate other than Roche lobe overflow, \emph{e.g.}\ wind. [$M_\odot$ yr$^{-1}$]
\item[tKh:] Kelvin-Helmholtz timescale
\item[LHe:] Luminosity due to helium burning
\item[RCZ:]
\item[McCO:] Mass of CO core
\item[TXH:] H abundance at $T_\mathrm{max}$
\item[TXHe:] He abundance at $T_\mathrm{max}$
\item[TXC:] C abundance at $T_\mathrm{max}$
\item[TXN:] N abundance at $T_\mathrm{max}$
\item[TXO:] O abundance at $T_\mathrm{max}$
\item[TXNe:] Ne abundance at $T_\mathrm{max}$
\item[TXMg:] Mg abundance at $T_\mathrm{max}$
\item[Tpsi:] Value of the electron degeneracy parameter at $T_\mathrm{max}$
\item[Trho:] log $T_\mathrm{max}$
\item[TT:] log rho at $T_\mathrm{max}$
\end{description}



\subsubsection*{Line 3:}
\begin{description}
\item[age:] Stellar age [yr]
\item[ecc:] Orbital eccentricity
\item[mdt:] Mass loss [$M_\odot$ yr$^{-1}$]
\item[tET:] Envelope Turnover timescale of the convective envelope
\item[LCO:] Luminosity due to Carbon/Oxygen burning
\item[DRCZ:]
\item[McNe:] Mass of Neon Core
\item[SXH:] Surface abundance of H
\item[SXHe:] Surface abundance of He
\item[SXC:] Surface abundance of C
\item[SXN:] Surface abundance of N
\item[SXO:] Surface abundance of O
\item[SXNe:] Surface abundance of Ne
\item[SXMg:] Surface abundance of Mg
\item[Spsi:] Surface value of the electron-degeneracy parameter
\item[Srho:] Log Surface density
\item[ST:] Log Surface Temperature
\end{description}



\subsubsection*{Line 4:}
\begin{description}
\item[cM:] Companion Mass [$M_\odot$]
\item[RLF1:] Relative Roche-lobe radius [$\log R/R_\mathrm{rlof}$]
\item[RLF2:] Relative Roche-lobe radius [$\log R/R_\mathrm{rlof}$]
\item[DLT:]
\item[Lnu:] Luminosity due to neutrino losses
\item[RA/R:] Alfv\'en radius [$R_*$]
\item[MH:] Total hydrogen mass in the star [$M_\odot$]
\item[conv. bdries:] Mass coordinates of convective boundaries (3 pairs)
\item[logR:] Log R
\item[logL:] log L
\end{description}



\subsubsection*{Line 5:}
\begin{description}

\item[Horb:] Orbital angular momentum
\item[F1:]
\item[DF21:]
\item[BE:]
\item[Lth:] Luminosity from contraction/expansion
\item[Bp:] Poloidal component of the magnetic field
\item[MHe:] Total helium mass in the star [$M_\odot$]
\item[semiconv. bdries:] Mass coordinates of semiconvective boundaries (3 pairs)
\item[k**2:] Dimensionless axis of gyration, if moment of inertia is calculated in the code.
\end{description}




\subsection{Convergence info}

\begin{description}
\item[Iter] The first integer displays the number of iterations,
\item[Err] The logarithm of the total error,
\item[Ferr] The residue in the current iteration,
\item[Fac] The factor by which corrections are multiplied before being applied. Normally 1.00, 
  but may be smaller if the code has trouble converging.
\end{description}

\noindent
Then a list of numbers follows, in pairs of an integer and a float (\emph{e.g.}\ $79-9.2$). There is one 
pair for each independent variable. The integer indicates the mesh point in the star ($1$ indicates the \textit{surface})
%(and \texttt{KH} the core) 
where the largest error for this independent variable occurs and the float indicates the log of the error 
in that mesh point.  In practice this means that $(199-9.9)$ is a good thing since $10^{-9.9}$ is a 
very small error, $(98-3.1)$ is worrying and when the floats get to $-2.0$ or larger something is 
really wrong.
It is usually a good idea to scroll up and look whether earlier blocks exist and, if so, to see whether 
the same variables are causing the problems there --- sometimes one variable starts causing problems and 
then drags along others.



%83-MV; -Not actually in .out1, it is in .out
%84- B-V; - Not actually in .out1, it is in .out
%*85- U-B; - Not actually in .out1, it is in .out 